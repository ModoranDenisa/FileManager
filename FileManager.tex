\documentclass[a4paper]{article}

%% Language and font encodings
\usepackage[english]{babel}
\usepackage[utf8x]{inputenc}
\usepackage[T1]{fontenc}

%% Sets page size and margins
\usepackage[a4paper,top=3cm,bottom=2cm,left=3cm,right=3cm,marginparwidth=1.75cm]{geometry}

%% Useful packages
\usepackage{amsmath}
\usepackage{graphicx}
\usepackage[colorinlistoftodos]{todonotes}
\usepackage[colorlinks=true, allcolors=blue]{hyperref}

\title{Manager pentru fisierele personale}
\author{Modoran Denisa Lorena}

\begin{document}
\maketitle

\begin{abstract}
Licenta mea este o aplicatie menita sa ajute fiecare utilizator sa isi tina in ordine si in siguranta fisierele personale.

\end{abstract}

\section{Introducere}

Aplicatia va fi de ajutor fiecarui om care va alege sa o foloseasca pentru ca e simplu de utilizat, nu necesita mult timp pentru a te obisnui sa lucrezi cu ea si este sigura pentru datele utilizatorilor.

\section{Utilizarea aplicatiei}

\subsection{Logarea clientilor}

La acest pas cei care doresc sa foloseasca aplicatia trebuie sa isi faca un cont daca nu au deja unul, in acest caz trebuie sa foloseasca o adresa de email valida, iar pe aceasta adresa vor primi un email de confirmare pentru a-si activa contul.

In cazul in care utilizatorul are deja un cont activ, se va loga cu adresa de email si paola setata la inceput.


\subsection{Crearea categoriilor de fisiere care se doresc a fi stocate}

Dupa crearea contului, fiecare utilizator este pregatit sa inceapa sa foloseasca aplicatia.

La acest pas utilizatorul isi creaza principalele categorii de fisiere pe care crede ca va dori sa le stocheze.

\subsection{Adaugarea propriu zisa de fisiere}

Dupa crearea categoriilor, utilizatorul poate intra in folderul corespunzator fisierului care se doreste sa fie stocat, selecteaza fisierul din laptopul personal si il incarca.

\subsection{Descarcarea fisierelor stocate in aplicatie}

Utilizatorii isi pot descarca in orice moment si pe orice device fisiere stocate in aplicatie, acestea vor fi descarcate sub forma de .zip si pot fi descarcate toate datele sau doar o parte din ele la alegere.






\end{document}